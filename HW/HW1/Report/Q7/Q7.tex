\section{Question 7}
\begin{proof}
    To prove that the inverse of a nonsingular diagonal matrix $D = [d_{ij}]$ is given by $D^{-1} = [1/\delta_{ij}] \cdot d_i$, where $\delta_{ij}$ is a function that is 1 if the variables $i$ and $j$ are equal, and 0 otherwise, we can follow these steps:
    
    1. First, let's establish that the product of $D$ and $D^{-1}$ results in the identity matrix $I$:
    
    \[
    DD^{-1} = I
    \]
    
    2. Now, we'll compute the product $DD^{-1}$:
    
    \[
    DD^{-1} = [d_{ij}][1/\delta_{ij}] \cdot d_i
    \]
    
    3. The product of $d_{ij}$ and $1/\delta_{ij}$ will be 1 if $i = j$ and 0 otherwise, according to the definition of $\delta_{ij}$. So, we have:
    
    \[
    DD^{-1} = [d_{ij}][1/\delta_{ij}] \cdot d_i = \delta_{ij} \cdot d_i = \begin{cases}
    d_i, & \text{if } i = j \\
    0, & \text{if } i \neq j
    \end{cases}
    \]
    
    4. Now, we can observe that the result is a diagonal matrix where all the diagonal elements are given by $d_i$ when $i = j$, and 0 when $i \neq j$. This is the definition of a diagonal matrix. So, $DD^{-1}$ is indeed a diagonal matrix.
    
    5. To determine the specific values on the diagonal, we can express the $DD^{-1}$ matrix explicitly:
    
    \[
    DD^{-1} = \begin{bmatrix}
    d_1 & 0 & 0 & \cdots & 0 \\
    0 & d_2 & 0 & \cdots & 0 \\
    0 & 0 & d_3 & \cdots & 0 \\
    \vdots & \vdots & \vdots & \ddots & \vdots \\
    0 & 0 & 0 & \cdots & d_n
    \end{bmatrix}
    \]
    
    6. We have now shown that $DD^{-1}$ is a diagonal matrix with the diagonal elements being the values from the original diagonal matrix $D$.
    
    7. For $DD^{-1}$ to be equal to the identity matrix $I$, all diagonal elements must be 1. Therefore, $d_i$ must be equal to 1 for all $i$.
    
    8. Thus, $D^{-1}$ is given by $D^{-1} = [1/\delta_{ij}] \cdot d_i$, where $\delta_{ij}$ is the function you described that is 1 if $i = j$ and 0 otherwise.
    
    This completes the proof that the inverse of a nonsingular diagonal matrix $D$ is given by $D^{-1} = [1/\delta_{ij}] \cdot d_i$ under the specified conditions.
    \end{proof}
    